\section{Experimentos Computacionais}\label{sec:experimentos} 
Os experimentos computacionais foram executados em uma máquina Intel Dual-Core de 2.81 GHz de clock e
2GB de memória RAM, rodando o sistema operacional Linux. O modelo matemático composto pela função objetivo \eqref{eq:objetivo1} e
as restrições \eqref{eq:ax1}-\eqref{eq:binarias} foi implementado
no Ilog CPLEX 12.5.1 e o algoritmo da seção \ref{sec:algoritmo} foi implementado
em Python 2.7, sendo que o otimizador utilizado para resolver o problema lagrangeano $\theta (\pi)$ presente nesse algoritmo foi o Ilog CPLEX 12.5.1.
Foram utilizados quatro conjuntos de instâncias de testes nos experimentos computacionais e essas instâncias foram retiradas de \cite{Beasley90}.
Nessas instâncias de testes cada linha da matriz de incidência $A$ é coberta por pelo menos duas colunas e cada coluna cobre pelo menos uma linha. O
custo $c_j$ de cada coluna $j$ está entre $[1,100]$. A tabela \ref{table:instancias} resume esses conjuntos de instâncias. A coluna 1 dessa tabela 
representa o identificador do conjunto da instância de teste,
as colunas 2 e 3 mostram, respectivamente, o número $m$ de linhas e $n$ de colunas da matriz de incidência $A$. A coluna 4 representa
a densidade da matriz $A$ que é calculado pelo quantidade de 1's dessa matriz dividido pela quantidade total de elementos de $A$ que é igual a 
$mn$ e a coluna 5 mostra a quantidade de problemas em cada conjunto. Os dez problemas do conjunto 4 são nomeados como scp41-scp410, os cinco
problemas do conjunto 6 são nomeados como scp61-scp65, e os problemas do conjunto A e B são nomeados respectivamente como scpa1-scpa5
e scpb1-scpb5.

\begin{table}[htbp]
\begin{center}
  \begin{tabular}{|c|r|r|r|r|}
    \hline
      Conjunto & Linhas   & Colunas & Densidade   & Problemas\\ \hline
      4        & 200      & 1000    & 2           & 10 \\ \hline
      6        & 200      & 1000    & 5           & 5 \\ \hline
      A        & 300      & 3000    & 2           & 5 \\ \hline
      B        & 300      & 3000    & 5           & 5 \\ \hline
  \end{tabular}
\caption{Detalhes das instâncias de testes utilizadas}
\label{table:instancias}
\end{center}
\end{table}
No experimento desse trabalho foi comparado a performance do modelo matemático para o $SCP$ que será 
chamado aqui de $IP$, com o algoritmo proposto na seção \ref{sec:algoritmo}, que será chamado $VolumeRevisado$.
O modelo $IP$ foi executado através do CPLEX com todos os parâmetros default. O modelo presente no algoritmo $VolumeRevisado$
foi executado pelo CPLEX também com os valores default. O algoritmo $VolumeRevisado$ foi executado com os parâmetros 
$m_1 = 0.0001$, $\delta_{min} = 0.00001$, com um número máximo de iterações igual a 500000. O modelo $IP$ e o algoritmo $VolumeRevisado$ foram executados com um tempo de execução máximo de 7200 segundos. \\
A tabela \ref{table:resultados4e6} apresenta os resultados obtidos para o conjuntos de instância 4 e 6 e a tabela
\ref{table:resultadosaeb} apresenta os resultados para o conjuntos de instância A e B. Nessas tabelas a
coluna 1 mostra o nome da instância de teste, as colunas 2 e 3 são resultados referentes ao modelo $IP$ e as colunas
4,5 e 6 são resultados referentes ao algoritmo $VolumeRevisado$. A coluna 2 apresenta o custo da solução obtido pela
modelo $IP$ e a coluna 3 apresenta o tempo consumido para encontrar essa solução. A coluna 4 apresenta o custo da solução 
obtido pelo algoritmo $VolumeRevisado$, a coluna 5 traz o número de iterações do algoritmo
e a coluna 6 mostra o tempo consumido pelo algoritmo $VolumeRevisado$.\\
Para todas as instâncias do conjunto 4 e 6 o CPLEX conseguiu encontrar soluções ótimas em um tempo muito pequeno, conforme
pode ser observado pelas colunas 2 e 3 da tabela \ref{table:resultados4e6}. Para as instâncias scp41, scp46 e scp410 o algoritmo
$VolumeRevisado$ parou com uma solução de custo muito próxima da solução do modelo matemático, conforme
pode ser observado na coluna 4, linhas 1,6 e 10 da tabela \ref{table:resultados4e6}.

\begin{table}[htbp]
\begin{center}
  \begin{tabular}{|c|r|r|r|r|r|}
    \hline
      Instância & \multicolumn{2}{|c|}{$IP$} & \multicolumn{3}{|c|}{$VolumeRevisado$}\\
                & Custo Solução    & Tempo(s)  & Custo Solução   & \#Iterações & Tempo(s)      \\ \hline
      scp41     & 429              & 0.84      & 428.21          & 38       & 7.84       \\ \hline
      scp42     & 512              & 0.85      & 492.33          & 35436    & 7200.00       \\ \hline
      scp43     & 516              & 0.86      & 479.98          & 34061    & 7200.00       \\ \hline
      scp44     & 494              & 0.86      & 480.21          & 34520    & 7200.00     \\ \hline
      scp45     & 512              & 0.85      & 499.30          & 35192    & 7200.00       \\ \hline
      scp46     & 560              & 0.89      & 559.10          & 41       & 8.24    \\ \hline
      scp47     & 430              & 0.83      & 295.89          & 34368    & 7200.00       \\ \hline
      scp48     & 492              & 0.97      & 467.75          & 34983    & 7200.00    \\ \hline
      scp49     & 641              & 0.90      & 443.61          & 34813    & 7200.00    \\ \hline
      scp410    & 514              & 0.90      & 513.06          & 31       & 6.31    \\ \hline
      scp61     & 138              & 1.23      & 105.98          & 34797    & 7200.00    \\ \hline
      scp62     & 146              & 2.06      & 76.75           & 35879    & 7200.00    \\ \hline
      scp63     & 145              & 1.26      & 83.10           & 35389    & 7200.00    \\ \hline
      scp64     & 131              & 0.96      & 127.38          & 35493    & 7200.00    \\ \hline
      scp65     & 161              & 1.94      & 118.19          & 35540    & 7200.00    \\ \hline
  \end{tabular}
\caption{Comparação entre os custos da solução e tempos obtidos entre o modelo $IP$ e o algoritmo $VolumeRevisado$ para as instâncias do conjunto 4 e 6.}
\label{table:resultados4e6}
\end{center}
\end{table}

Para todas as instâncias do conjunto A e B o CPLEX conseguiu encontrar soluções ótimas em um tempo pequeno, conforme
pode ser observado pelas colunas 2 e 3 da tabela \ref{table:resultadosaeb}. Para essas instâncias o algoritmo {$VolumeRevisado$}
não conseguiu encontrar a solução ótima ou próxima da ótima para nenhuma delas, conforme pode ser observado pela coluna 4 da 
tabela \ref{table:resultadosaeb}.
\begin{table}[htbp]
\begin{center}
  \begin{tabular}{|c|r|r|r|r|r|}
    \hline
      Instância & \multicolumn{2}{|c|}{$IP$} & \multicolumn{3}{|c|}{$VolumeRevisado$}\\
                & Custo Solução    & Tempo(s)  & Custo Solução   & \#Iterações & Tempo(s)      \\ \hline
      scpa1     & 253              & 9.95      & 153.88          & 9859     & 7200.00  \\ \hline
      scpa2     & 252              & 9.87      & 209.80          & 10010    & 7200.00  \\ \hline
      scpa3     & 232              & 9.48      & 208.89          & 9804     & 7200.00 \\ \hline
      scpa4     & 234              & 8.76      & 207.49          & 10025    & 7200.00  \\ \hline
      scpa5     & 236              & 8.64      & 186.64          & 10006    & 7200.00 \\ \hline
      scpb1     & 69               & 10.08     & 55.89           & 10078    & 7200.00  \\ \hline
      scpb2     & 76               & 10.87     & 45.80           & 9867     & 7200.00  \\ \hline
      scpb3     & 80               & 9.67      & 55.92           & 9820     & 7200.00  \\ \hline
      scpb4     & 79               & 11.52     & 66.59           & 10024    & 7200.00  \\ \hline
      scpb5     & 72               & 9.86      & 53.16           & 10012    & 7200.00  \\ \hline
  \end{tabular}
\caption{Comparação entre os custos da solução e tempos obtidos entre o modelo $IP$ e o algoritmo $VolumeRevisado$ para as instâncias do conjunto A e B.}
\label{table:resultadosaeb}
\end{center}
\end{table}












